% ---
% Dedicatória
% ---
\begin{dedicatoria}
   \vspace*{\fill}
   \centering
   \noindent
   \textit{ Este trabalho é dedicado às crianças adultas que,\\
     quando pequenas, sonharam em se tornar cientistas.}
   \vspace*{\fill}
\end{dedicatoria}
% ---

% ---
% Agradecimentos
% ---
\begin{agradecimentos}
  % Lembrar de agradecer à agência de fomento, em caso de bolsa.
  \lipsum[1]

\end{agradecimentos}
% ---

% ---
% Epígrafe
% ---
\begin{epigrafe}
  \vspace*{\fill}
  \begin{flushright}
    \textit{``Não vos amoldeis às estruturas deste mundo, \\
      mas transformai-vos pela renovação da mente, \\
      a fim de distinguir qual é a vontade de Deus: \\
      o que é bom, o que Lhe é agradável, o que é perfeito.\\
      (Bíblia Sagrada, Romanos 12, 2)}
  \end{flushright}
\end{epigrafe}
% ---

% ---
% RESUMOS
% ---

% resumo em português
\setlength{\absparsep}{18pt} % ajusta o espaçamento dos parágrafos do resumo
\begin{resumo}
  Segundo a \citeonline[3.1-3.2]{NBR6028:2003}, o resumo deve
  ressaltar o objetivo, o método, os resultados e as conclusões do
  documento. A ordem e a extensão destes itens dependem do tipo de
  resumo (informativo ou indicativo) e do tratamento que cada item
  recebe no documento original. O resumo deve ser precedido da
  referência do documento, com exceção do resumo inserido no próprio
  documento. (\ldots) As palavras-chave devem figurar logo abaixo do
  resumo, antecedidas da expressão Palavras-chave:, separadas entre si
  por ponto e finalizadas também por ponto.

 \textbf{Palavras-chave}: latex. abntex. editoração de texto.
\end{resumo}

% resumo em inglês
\begin{resumo}[Abstract]
  \begin{otherlanguage*}{english}
    This is the english abstract.

    \vspace{\onelineskip}

    \noindent
    \textbf{Keywords}: latex. abntex. text editoration.
  \end{otherlanguage*}
\end{resumo}

% resumo em francês
\begin{resumo}[Résumé]
  \begin{otherlanguage*}{french}
    Il s'agit d'un résumé en français.

    \textbf{Mots-clés}: latex. abntex. publication de textes.
  \end{otherlanguage*}
\end{resumo}

% resumo em espanhol
\begin{resumo}[Resumen]
  \begin{otherlanguage*}{spanish}
    Este es el resumen en español.

    \textbf{Palabras clave}: latex. abntex. publicación de textos.
  \end{otherlanguage*}
\end{resumo}
% ---
